\documentclass{article}
\usepackage[utf8]{inputenc}

\title{CLOSURE COMPILER}
\author{bamanyeismail }
\date{March 2018}
\begin{document}
\maketitle
\section{Closure Compiler?}
The Closure Compiler is a tool for making JavaScript download and run faster. Instead of compiling from a source language to machine code, it compiles from JavaScript to better JavaScript. It parses your JavaScript, analyzes it, removes dead code and rewrites and minimizes what's left. It also checks syntax, variable references, and types, and warns about common JavaScript pitfalls.
\section{How it is used;}
You can use the Closure Compiler as:
\newline•  An open source Java application that you can run from the command line.
\newline•  A simple web application.
\newline•  A RESTful API.
\section{Benefits of using Closure Compiler?}
•	Efficiency. The Closure Compiler reduces the size of your JavaScript files and makes them more efficient, helping your application to load faster and reducing your bandwidth needs.
\newline•	Code checking. The Closure Compiler provides warnings for illegal JavaScript and warnings for potentially dangerous operations, helping you to produce JavaScript that is less buggy and easier to maintain.
       \section{Some of the questions asked by the Closure Compiler User}
•	How does the Closure Compiler work with the Closure Library?
\newline•	The Closure Compiler provides special checks and optimizations for code that uses the Closure Library. In addition, the Closure Compiler service can automatically include Closure Library files. Finding Your Way around Closure describes the syntax for declaring the parts of Closure that you need. See the API referencefor information on using the Closure Library with the API. To use the Closure Library with the Closure Compiler application you must first download the Closure Library. Support for the Closure Library is enabled in the compiler application by default.

\section{References}
[1]"Closure Compiler  |  Google Developers", Google Developers, 2018. [Online]. Available: https://developers.google.com/closure/compiler/. [Accessed: 09- Mar- 2018].


\end{document}
